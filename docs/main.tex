
\documentclass{article}
\usepackage{amsmath,amssymb}
\usepackage{graphicx}
\usepackage{makeidx}
\usepackage{algpseudocode}
\usepackage{algorithm}
\usepackage{listing}
\evensidemargin=0.20in
\oddsidemargin=0.20in
\topmargin=0.2in
%\headheight=0.0in
%\headsep=0.0in
%\setlength{\parskip}{0mm}
%\setlength{\parindent}{4mm}
\setlength{\textwidth}{6.4in}
\setlength{\textheight}{8.5in}
%\leftmargin -2in
%\setlength{\rightmargin}{-2in}
%\usepackage{epsf}
%\usepackage{url}

\usepackage{booktabs}   %% For formal tables:
                        %% http://ctan.org/pkg/booktabs
\usepackage{subcaption} %% For complex figures with subfigures/subcaptions
                        %% http://ctan.org/pkg/subcaption
\usepackage{enumitem}
%\usepackage{minted}
%\newminted{fortran}{fontsize=\footnotesize}

\usepackage{xargs}
\usepackage[colorinlistoftodos,prependcaption,textsize=tiny]{todonotes}

\usepackage{hyperref}
\newcommand{\N}{\mathbb{N}}
\newcommand{\Z}{\mathbb{Z}}

\newcommand{\val}{\mathbb{V}}
\newcommand{\chunkid}{\texttt{chunkid}}
\newcommand{\chunk}{\texttt{chunk}}
\newcommand{\mem}{\texttt{mem}}
\newcommand{\scop}{\texttt{SCoP}}
\newcommand{\scops}{\scop s}
\newcommand{\memacc}{\texttt{memacc}}
\newcommand{\MEMACC}{\texttt{MEMACC}}
\newcommand{\timepoint}{\texttt{timepoint}}
\newcommand{\iterspace}{\texttt{iterspace}}
\newcommand{\arrspace}{\texttt{arrspace}}
\newcommand{\readmemacc}{\texttt{readmemacc}}
\newcommand{\writememacc}{\texttt{writememacc}}
\newcommand{\schedule}{\texttt{schedule}}
\newcommand{\readtag}{\texttt{tag}}
\newcommand{\scatterspace}{\texttt{scatterspace}}
\newcommand{\vivspace}{\texttt{vivspace}}
\newcommand{\stmt}{\texttt{stmt}}


\title{Polyhedral compilation formalism}
\author{Siddharth Bhat}

\begin{document}
\maketitle
\date{}
\tableofcontents

\section{Memory}
A chunk (\chunk) is conceptually contiguous section of memory, usually associated
with that of an array or a struct. It is mathematically represented as 
a mapping from naturals to a value domain, $\val$.

Memory (\mem) is conceptually a mapping of chunk-IDs (\chunkid) to chunks, where each chunk-ID
is mathematically represented by a natural number.


\begin{align*}
    \chunk &: \N \to \val \\ 
    \mem &: \text{\chunkid} \to \chunk
\end{align*}

Two chunks are considered to \textit{never alias}. Two memory accesses alias
iff they access the same chunk and the same index within the chunk.

\section{\scops}
A \scop (Static Control Part) is mathematical representation of programs
which have control-flow that can be analyzed statically.

For this model, we require a notion of:

\subsection{\memacc}
A memory access is a mapping from a \timepoint~to an array index. More formally,
it maps points in the iteration space of the scop to points in the array space
of a given array.

A memory access can either be a read access or a write access.

\subsubsection{Read accesses}
\begin{align*}
    &\readmemacc(S, A) \equiv  (tag: \readtag) \times (ixfn: \iterspace(S) \to \arrspace(A)) \\
\end{align*}

Every read access has a $tag$, telling the abstract name of the read value,
an $ixfn$, telling the index of the array which is read from at a given timepoint.

\subsubsection{Write accesses}
\begin{align*}
    &\writememacc(S, A) \equiv  (valfn: \iterspace(S) \times (\readtag \to \val)  \to \val) \times (ixfn: \iterspace(S) \to \arrspace(A))
\end{align*}
Every write acces has a $valfn$, which given the timepoint and all the previous
read values computes the value to write, and an $ixfn$, telling the index
of the array which is written to at a given timepoint.


\subsection{\schedule}
\begin{align*}
\vivspace \subset \Z^n \\
\scatterspace \subseteq \Z^m \\
\schedule : \vivspace \to \scatterspace \\
\end{align*}

A schedule is a mapping of virtual induction variables ($\vivspace \subset \Z^n$)
into points in the scatter space ($\scatterspace \subseteq \Z^n$). In general, any polyhedral object is
said to be \textit{scheduled} if it is mapped into the scatter space, and
is arranged according to lexicographic-ascending order in this scatter space.

In this formalism, we \textit{schedule statements}. 

\subsection{\stmt}
A scop statement is a collection of \readmemacc and \writememacc,
where intuitively, all the \textbf{reads happen atomatically}, sequenced by 
\textbf{all writes happening atomically}. Hence, all \textit{writes can commute} in a statement,
with semantic preservation.

The atomicity property is useful, since it allows us to not represent certain
dependences between multiple reads or multiple writes, which we would
otherwise be forced to do.

Intuitively, a scop statement represents a "maximum chunk of atomic memory
modifications" which can occur together.

\end{document}
